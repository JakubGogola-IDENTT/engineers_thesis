\chapter{Zawartość płyty CD}
\thispagestyle{chapterBeginStyle}
\label{plytaCD}

Płyta CD dołączona do niniejszej pracy zawiera następujące elementy:
\begin{enumerate}
    \item Kody źródłowe pisemnej części pracy w języku \LaTeX wraz z niezbędnymi grafikami.
    \item Skompilowany ze źródeł plik PDF z pracą.
    \item Kody źródłowe części implantacyjnej w języku \texttt{Go}.
\end{enumerate}

Struktura katalogów na płycie CD została przedstawiona na \ref{fig:dir_structure}.

\begin{figure}[h]
    \centering
    \framebox[\textwidth]{%
        \begin{minipage}{0.9\textwidth}
          \dirtree{%
            .1 /latex.
            .1 /src.
                .2 /genetics.
                .2 /cofig.
            .1 /pdf.
            }
        \end{minipage}
    }
    \caption{Struktura plików}
    \label{fig:dir_structure}
\end{figure}
W katalog \texttt{/latex} znajdują się kody źródłowe pisemnej części pracy, w katalogu \texttt{/pdf} znajduje się skompilowany plik PDF, a w \texttt{/src} pliki źródłowe części implementacyjnej.
